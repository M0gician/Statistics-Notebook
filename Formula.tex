\documentclass[Main.tex]{subfiles}

\begin{document}
	%-=-=-=-=-=-=-=-=-=-=-=-=-=-=-=-=-=-=-=-=-=-=-=-=
	%
	%	CHAPTER
	%
	%-=-=-=-=-=-=-=-=-=-=-=-=-=-=-=-=-=-=-=-=-=-=-=-=
	
	\chapter{Formulae}
	
	%-=-=-=-=-=-=-=-=-=-=-=-=-=-=-=-=-=-=-=-=-=-=-=-=
	%	SECTION:
	%-=-=-=-=-=-=-=-=-=-=-=-=-=-=-=-=-=-=-=-=-=-=-=-=
	
	\begin{exercise}[Mean and median]\index{Mean and median} \hfill \\
		
		\begin{itemize}
			\item The \textbf{Mean} is the arithmetic average of a set of number.
			\begin{definition}[Mean]\index{Mean}				
				\begin{subequations}
					\begin{align}
						\bar{x}=\frac{\text{sum of observations}}{n}=\frac{x_{1}+x_{2}+\cdot +x_{n}}{n}=\frac{\sum x_{i}}{n}
					\end{align}
				\end{subequations}
			\end{definition} \hfill
			
			\item The \textbf{Median} is the midpoint of a distribution, the number such that half the observations are smaller and half are larger.\\
			
			\item \textbf{Remember}: the \underline{median} is a \textbf{resistant} measure if center because it is relatively unaffected by extreme observations. The \underline{mean} is \textbf{nonresistant}. Among the measures of spread, the \emph{\underline{IQR}} is \textbf{resistant}, but the \underline{standard deviation} and \underline{range} are \textbf{nonresistant}.
		\end{itemize}	
	\end{exercise}
	
	\begin{exercise}[Measuring Spread]\index{Measuring Spread} \hfill \\
		
		\begin{itemize}
			\item \textbf{Interquartile range} (IQR)
			\begin{definition}[Interquartile range (IQR)]\index{Interquartile range (IQR)}			
				\begin{subequations}
					\begin{align}
						IQR=Q_{3}-Q_{1}
					\end{align}
				\end{subequations}
			\end{definition}\hfill
			
			\item \textbf{Outliers}
			\begin{definition}[Outlier]\index{Outlier}
				\begin{subequations}
					\begin{align}
						\text{Outlier}
						\begin{cases}
							>Q_{3}+(1.5\times IQR)\\
							<Q_{1}-(1.5\times IQR)
						\end{cases}
					\end{align}
				\end{subequations}
			\end{definition}\hfill
			
			\item \textbf{Standard Deviation}  $s_{x}$ measures the typical distance of the values in  a distribution from the mean.
			\begin{definition}[Standard Deviation]\index{Standard Deviation}			
				\begin{subequations}
					\begin{align}
						S_{x}=\sqrt{\frac{1}{n-1}\sum(x_{i}-\bar{x})^{2}}
					\end{align}
				\end{subequations}
			\end{definition}\hfill
			
			\item \textbf{Variance} $s^{2}_{x}$  is the average squared deviation of a set of number.
			\begin{definition}[Variance]\index{Variance}			
				\begin{subequations}
					\begin{align}
						S^{2}_{x}=\frac{(x_{1}-\bar{x})^{2}+(x_{2}-\bar{x})^{2}+\cdots+(x_{n}-\bar{x})^{2}}{n-1}=\frac{1}{n-1}\sum(x_{i}-\bar{x})^{2}
					\end{align}
				\end{subequations}
			\end{definition}\hfill
		\end{itemize}						
	\end{exercise}
	
	\section{Describing Location in a Distribution}\index{Describing Location in a Distribution}
	
	\begin{example}[Measuring Positions]\index{Measuring Positions} \hfill \\
		\begin{itemize}		
			\item \textbf{Standardized Score} ($\mathbb{Z}$-Score): if $x$ is an observation from a distribution that has known mean and standard deviation, the \textbf{standardized score} for $x$ is 
			
			\begin{definition}[Normal Distribution]\index{Normal Distribution}				
				\begin{subequations}
					\begin{align}
						\mathbb{Z}=\frac{x_{i}-\mu}{\sigma}
					\end{align}
				\end{subequations}
			\end{definition}\hfill 
		\end{itemize}	
	\end{example}
	
	\begin{exercise}[Normal Distribution]\index{Normal Distribution} \hfill \\
		\begin{itemize}
			\item A \textbf{Normal Distribution} is described by a Normal density curve.\\ The \textbf{mean} of a Normal distribution $\mu$ is at the center of the symmetric \textbf{Normal curve}.\\ The \textbf{standard deviation} $\sigma$ is the distance from the center to the change-of-curvature points on either side.\\
			
			\item We \textbf{abbreviate} the Normal distribution with mean $\mu$ and standard deviation $\sigma$ as $N(\mu,\sigma)$.
			
			\begin{definition}[Normal Distribution]\index{Normal Distribution}			
				\begin{subequations}
					\begin{align}
						f(x|\mu,\sigma)=\frac{1}{\sigma\sqrt{2\pi}}e^{-\frac{(x-\mu)^{2}}{2\sigma^{2}}}
					\end{align}
				\end{subequations}	
			\end{definition}\hfill 
			
			\item The \textbf{standard Noraml distribution} is the Normal distribution with mean 0 and standard deviation 1.
			
			\begin{definition}[Standard Normal Distribution]\index{Normal Distribution}			
				\begin{subequations}
					\begin{align}
						\mathbb{Z}=\frac{x_{i}-\mu}{\sigma}\Rightarrow\text{Standar}&\text{dization}=\frac{\text{Obs}-\text{Mean}}{\text{SD}}\\
						\mathit{f}(x|0,1)&=\frac{1}{\sqrt{2\pi}}e^{-\frac{1}{2}x^{2}} \\
						\int^{\infty}_{-\infty}\frac{1}{\sqrt{2\pi}}&e^{-\frac{1}{2}x^{2}}\mathrm{d}x=1 
					\end{align}
				\end{subequations}	
			\end{definition}\hfill								
		\end{itemize}	
	\end{exercise}
	
	\begin{example}[Measuring Linear Association]\index{Measuring Linear Association} \hfill \\
		\begin{itemize}	
			\item Correlation\hfill
			\begin{definition}[Correlation]\index{Correlation}
				\begin{subequations}
					\begin{align}
						r&=\frac{Z_{x_{1}}Z_{y_{1}}+Z_{x_{2}}Z_{y_{2}}+\cdots+Z_{x_{n}}Z_{y_{n}}}{n-1}\\
						&=\frac{1}{n-1}\sum^{n}_{i=1}(\frac{x_{i}-\bar{x}}{S_{x}})(\frac{y_{i}-\bar{y}}{S_{y}})\\
						&=\frac{1}{n-1}\sum Z_{x}Z_{y} \qquad (Z=\frac{x_{i}-\bar{x}}{S_{x}})	\hfill	
					\end{align}
				\end{subequations}				
			\end{definition}\hfill 		
		\end{itemize}	
	\end{example}\hfill	
	
	\begin{exercise}[The Role of $s$ and $r^{2}$ in Regression]\index{The Role of $s$ and $r^{2}$ in Regression} \hfill \\
		\begin{itemize}	
			\item If we use a least-squares line to predict the values of a response variable $y$ from an explanatory variable $x$, the \textbf{standard deviation of the residuals} $s$ is
			
			\begin{definition}[Standard deviation of the residuals $s$]\index{Standard deviation of the residuals $s$}
				\begin{subequations}
					\begin{align}
						s=\sqrt{\frac{\sum\text{residuals}^{2}}{n-2}}=\sqrt{\frac{\sum(y_{i}-\hat{y})^{2}}{n-2}}
					\end{align}
				\end{subequations}
			\end{definition} \hfill
			
			This value gives the approximate size of a typical \textbf{prediction error} (residual)\\
			
			\item The \textbf{coefficient of determination} $r^{2}$ is the fraction of the variation in the values of $y$ that is accounted for by the least-squares regression line of $y$ on $x$.\hfill \\
			(The \textbf{percentage} of how well the line fits those data)
			
			\begin{definition}[Coefficient of determination  $r^{2}$]\index{Coefficient of determination $r^{2}$}
				\begin{subequations}
					\begin{align}
						r^{2}=1-\frac{\sum\text{residuals}^{2}}{\sum(y_{i}-\bar{y})}=1-\frac{\sum(\text{Obs$-$prediction})^{2}}{\sum(\text{Obs$-$mean Obs})^{2}}
					\end{align}
				\end{subequations}
			\end{definition} \hfill						
		\end{itemize}
	\end{exercise}
	
	\section{Probability Rules}\index{Probability Rules}	
	\begin{example}[General Addition Rule For Two Events]\index{General Addition Rule For Two Events} \hfill \\
		\begin{itemize}	
			\item If $A$ and $B$ are any two events resulting from some chance process, the
			\begin{definition}[General Addition Rule For Two Events]\index{General Addition Rule For Two Events}
				\begin{subequations}
					\begin{align}
						P(A\cup B)=P(A)+P(B)-P(A\cap B)	
					\end{align}
				\end{subequations}
			\end{definition}\hfill									
		\end{itemize}
	\end{example}												
	
	\begin{example}[General Addition Rule For Two Events]\index{General Addition Rule For Two Events} \hfill \\
		\begin{itemize}	
			\item If $A$ and $B$ are any two events resulting from some chance process, the
			\begin{definition}[General Addition Rule For Two Events]\index{General Addition Rule For Two Events}
				\begin{subequations}
					\begin{align}
						P(A\cup B)=P(A)+P(B)-P(A\cap B)	
					\end{align}
				\end{subequations}
			\end{definition}\hfill									
		\end{itemize}
	\end{example}
	
	\begin{exercise}[Independent events]\index{Independent events} \hfill \\
		\begin{itemize}	
			\item When events $A$ and $B$ are \textbf{independent}:
			\begin{subequations}
				\begin{align}
					P(A|B)=P(A)\quad\text{and}\quad P(B|A)=P(B)
				\end{align}
			\end{subequations}	
			where $P(B|A)$ is the conditional probability that event $B$ occurs given that event $A$ has already occurred.\\
			\begin{definition}[Multiplication rule for independent events]\index{Multiplication rule for independent events}
				\begin{subequations}
					\begin{align}
						P(A\cap B)=P(A)\cdot P(B)
					\end{align}
				\end{subequations}	
			\end{definition} \hfill					
		\end{itemize}
	\end{exercise}
	
	\section{Conditional Probability and Independence}\index{Conditional Probability and Independence}			
	\begin{example}[Mean (Expected Value) of a Discrete Random Variable]\index{Mean (Expected Value) of a Discrete Random Variable} \hfill \\
		\begin{itemize}				
			\item Suppose that $X$ is a discrete random variable with probability distribution\hfill			 		
			\begin{table}[H]
				\centering
				\begin{tabular}{lcccr}
					\hline
					\textbf{Value:} & $x_{1}$ & $x_{2}$ & $x_{3}$ & $\cdots$ \\
					\textbf{Probability:} & $p_{1}$ & $p_{2}$ & $p_{3}$ & $\cdots$ \\
					\hline
				\end{tabular}
			\end{table}
			To find the \textbf{mean (expected value)} of $X$, multiply each possible value by its probability, the add all the products:\hfill \\ \hfill \\ \hfill \\ \hfill \\
			\begin{definition}[Mean (Expected Value)]\index{Mean (Expected Value)}
				\begin{subequations}
					\begin{align}
						\mu_{x}=E(X)&=x_{1}p_{1}+x_{2}p_{2}+x_{3}p_{3}+\cdots\\
						&=\sum x_{i}p_{i} \notag
					\end{align}
				\end{subequations}	
			\end{definition}\hfill			
		\end{itemize}
	\end{example}
	
	\section{Binomial and Geometric Random Variables}\index{Binomial and Geometric Random Variables}
	
	\begin{example}[Binomial Probabilities]\index{Binomial Probabilities} \hfill \\
		\begin{itemize}	
			\item The \textbf{Binomial Coefficient}\hfill
			\begin{definition}[Binomial Coefficient]\index{Binomial Coefficient}\hfill
				\begin{subequations}
					\begin{align}
						\binom{n}{k} =\frac{n!}{k!(n-k)!}
					\end{align}
				\end{subequations}\hfill	
			\end{definition}
			counts the number of ways $k$ successes can be arranged among $n$ trails. The \textbf{factorial} of $n$ is\hfill
			\begin{subequations}
				\begin{align}
					n!=n(n-1)(n-2)\cdots (3)(2)(1)
				\end{align}
			\end{subequations}	
			for positive whole numbers $n$, and $0!=1$\hfill \\
			\item \textbf{Binomial Probability Formula}
			\begin{definition}[Binomial Probability]\index{Binomial Probability}\hfill
				\begin{subequations}
					\begin{align}
						P(X=K)=\binom{n}{k}p^{k}(1-p)^{n-k}
					\end{align}
				\end{subequations}\hfill	
			\end{definition}
			If a count $X$ of successes has the binomial distribution with number of trials $n$ and probability of success $p$, the \textbf{mean} and \textbf{standard deviation} of $X$ are	
			\begin{definition}[Mean and Standard deviation of Binomial distribution]\index{[Mean and Standard deviation of Binomial distribution}\hfill
				\begin{subequations}
					\begin{align}
						\mu_{x}&=np\\
						\sigma_{x}&=\sqrt{np(1-p)}
					\end{align}
				\end{subequations}\hfill	
			\end{definition}												
		\end{itemize}
	\end{example}
	
	\begin{example}[Geometric Random Variables]\index{Geometric Random Variables} \hfill \\
		\begin{itemize}	
			\item A \textbf{Geometric setting} consists of repeated trails of the same chance process in which the probability $p$ of successes is the same on each trail, and the goal is to count the number of trails it takes to get one success.\hfill \\
			\item If $Y=$ the number of trails required to obtain the first success, then $Y$ is a \textbf{Geometric probability} that $Y$ takes any value is
			\begin{definition}[Geometric Probability]\index{Geometric Probability}\hfill
				\begin{subequations}
					\begin{align}
						P(Y=K)=(1-p)^{k-1}p
					\end{align}
				\end{subequations}\hfill	
			\end{definition}
			The \textbf{mean} (expected value) of a geometric random variable $Y$ is
			\begin{definition}[Mean of Geometric Probability]\index{Mean of Geometric Probability}\hfill
				\begin{subequations}
					\begin{align}
						\mu_{Y}=E(Y)=\frac{1}{p}
					\end{align}
				\end{subequations}\hfill	
			\end{definition}
			which is the expected number of trails required to get the first success.						
		\end{itemize}
	\end{example}													
\end{document}				